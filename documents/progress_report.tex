%%%%%%%%%%%%%%%%%%%%%%%%%%%%%%%%%%%%%%%%%%%%%%%%%%%%%%%%%%%%%%%%%%%%%%%%%%%%%%%%%%%%%%%%%%%%%%%%
%
% CSCI 1430 Written Question Template
%
% This is a LaTeX document. LaTeX is a markup language for producing documents.
% Your task is to answer the questions by filling out this document, then to
% compile this into a PDF document.
%
% TO COMPILE:
% > pdflatex thisfile.tex

% If you do not have LaTeX, your options are:
% - VSCode extension: https://marketplace.visualstudio.com/items?itemName=James-Yu.latex-workshop
% - Online Tool: https://www.overleaf.com/ - most LaTeX packages are pre-installed here (e.g., \usepackage{}).
% - Personal laptops (all common OS): http://www.latex-project.org/get/ 
%
% If you need help with LaTeX, please come to office hours.
% Or, there is plenty of help online:
% https://en.wikibooks.org/wiki/LaTeX
%
% Good luck!
% The CSCI 1430 staff
%
%%%%%%%%%%%%%%%%%%%%%%%%%%%%%%%%%%%%%%%%%%%%%%%%%%%%%%%%%%%%%%%%%%%%%%%%%%%%%%%%%%%%%%%%%%%%%%%%
%
% How to include two graphics on the same line:
% 
% \includegraphics[width=0.49\linewidth]{yourgraphic1.png}
% \includegraphics[width=0.49\linewidth]{yourgraphic2.png}
%
% How to include equations:
%
% \begin{equation}
% y = mx+c
% \end{equation}
% 
%%%%%%%%%%%%%%%%%%%%%%%%%%%%%%%%%%%%%%%%%%%%%%%%%%%%%%%%%%%%%%%%%%%%%%%%%%%%%%%%%%%%%%%%%%%%%%

\documentclass[11pt]{article}

\usepackage[english]{babel}
\usepackage[utf8]{inputenc}
\usepackage[colorlinks = true,
            linkcolor = blue,
            urlcolor  = blue]{hyperref}
\usepackage[a4paper,margin=1.5in]{geometry}
\usepackage{stackengine,graphicx}
\usepackage{fancyhdr}
\setlength{\headheight}{15pt}
\usepackage{microtype}
\usepackage{times}
\usepackage{booktabs}

% From https://ctan.org/pkg/matlab-prettifier
\usepackage[numbered,framed]{matlab-prettifier}

\frenchspacing
\setlength{\parindent}{0cm} % Default is 15pt.
\setlength{\parskip}{0.3cm plus1mm minus1mm}

\pagestyle{fancy}
\fancyhf{}
\lhead{Final Project Progress Report}
\rhead{CSCI 1430}
\rfoot{\thepage}

\date{}

\title{\vspace{-1cm}Final Project Progress Report}

\begin{document}
\maketitle
\vspace{-1cm}
\thispagestyle{fancy}
**Important**: In your report, please
1) Make it very clear who on the team contributed what, and 
2) Include the dates of at least *two* meetings you had with your mentor.
\textbf{Team name: \emph{A Fine Matrix}}\\
\textbf{TA name: \emph{Shania}}

\emph{Note:} when submitting this document to Gradescope, make sure to add all other team members to the submission. This can be done on the submission page after uploading.

\section*{Progress Report Instructions}

Before writing your progress report, you should have met with your TA and talked through your progress.

\textbf{Mentor TA Meeting:} 4/19/2024

\subsection*{Team contributions}

Please describe in one paragraph (3--4 sentences) per team member what each of you contributed to the project so far.
\begin{description}

\item[Person 1] So far, I have read several papers on possible model architectures we can use. The main papers were \href{http://arxiv.org/abs/2206.00311}{Mask R-CNN}, \href{https://arxiv.org/pdf/1704.03155.pdf}{EAST}, and one on the relevant \href{https://arxiv.org/pdf/1907.00945.pdf}{ICDAR 2019 Robust Reading Challenge}. I have also begun work on implementing the EAST detector in Tensorflow.

\item[Person 2] I have been investigating the Google Translate API and seeing if there any free ways to use it or other alternatives. I've also been figuring out what library/libraries to use for video display with text overlay. It looks like the Python OpenCV library is perfect for this.

\item [Person 3] I have been mainly focusing on finalizing the pipeline for our model; this can be separated into a model that finds the correct location of text in an image (the bounding boxes’ coordinates for each word), one that identifies that script/language of each word, one that uses the Google Translate API to translate the text as desired, and one that displays the output of the API on top of our original bounding boxes. I have found the appropriate dataset that should be used to train these different models within our overall architecture (e.g. the first model can be trained with the MLT-19 dataset). I am now currently working on preprocessing our datasets and implementing the first model (which is a Mask R-CNN).

\end{description}

\end{document}
